\documentclass{article}
\usepackage{minted}
\author{Reinaldy Rafli}
\date{\today}
\title{Bubble Sort Algorithm}
\begin{document}

\maketitle

The bubble sort works by passing sequentially over a list, comparing each value to the one immediately after it. 
If the first value is greater than the second, their positions are switched. Over a number of passes, at most equal
to the number of elements in the list, all of the values drift into their correct positions (large values "bubble"
rapidly toward the end, pushing others down around them). Because each pass finds the maximum item and puts it
at the end, the portion of the list to be sorted can be reduced at each pass. A boolean variable is used to track 
whether any changes have been made in the current pass; when a pass completes without changing anything, the algorithm exits.

Because of its abysmal O(n2) performance, it is not used often for large (or even medium-sized) datasets.

\begin{minted}{}
  repeat
      if itemCount <= 1
          return
      hasChanged := false
      decrement itemCount
      repeat with index from 1 to itemCount
          if (item at index) > (item at (index + 1))
              swap (item at index) with (item at (index + 1))
              hasChanged := true
  until hasChanged = false
\end{minted}

\begin{minted}{c}
  #include <stdio.h>
 
  void bubble_sort (int *a, int n) {
      int i, t, j = n, s = 1;
      while (s) {
          s = 0;
          for (i = 1; i < j; i++) {
              if (a[i] < a[i - 1]) {
                  t = a[i];
                  a[i] = a[i - 1];
                  a[i - 1] = t;
                  s = 1;
              }
          }
          j--;
      }
  }
    
  int main () {
      int a[] = {4, 65, 2, -31, 0, 99, 2, 83, 782, 1};
      int n = sizeof a / sizeof a[0];
      int i;
      for (i = 0; i < n; i++)
          printf("%d%s", a[i], i == n - 1 ? "\n" : " ");
      bubble_sort(a, n);
      for (i = 0; i < n; i++)
          printf("%d%s", a[i], i == n - 1 ? "\n" : " ");
      return 0;
  } 
\end{minted}

\begin{minted}{go}
  package main
 
  import "fmt"
  
  func main() {
      list := []int{31, 41, 59, 26, 53, 58, 97, 93, 23, 84}
      fmt.Println("unsorted:", list)
  
      bubblesort(list)
      fmt.Println("sorted!  ", list)
  }
  
  func bubblesort(a []int) {
      for itemCount := len(a) - 1; ; itemCount-- {
          hasChanged := false
          for index := 0; index < itemCount; index++ {
              if a[index] > a[index+1] {
                  a[index], a[index+1] = a[index+1], a[index]
                  hasChanged = true
              }
          }
          if hasChanged == false {
              break
          }
      }
  }
\end{minted}