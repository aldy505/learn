\documentclass{article}
\usepackage{minted}
\author{Reinaldy Rafli}
\date{\today}
\title{Bubble Sort Algorithm}
\begin{document}

\maketitle

The singleton pattern is one of the simplest design patterns. Sometimes we need to have only one instance of our class 
for example a single DB connection shared by multiple objects as creating a separate DB connection for every object 
may be costly. Similarly, there can be a single configuration manager or error manager in an application that handles all 
problems instead of creating multiple managers.

Singleton is a part of Gang of Four design pattern and it is categorized under creational design patterns. 
In this article, we are going to take a deeper look into the usage of the Singleton pattern. 
It is one of the most simple design patterns in terms of the modelling but on the other hand, this is one of the most 
controversial patterns in terms of complexity of usage.

Singleton pattern is a design pattern which restricts a class to instantiate its multiple objects. It is nothing but 
a way of defining a class. Class is defined in such a way that only one instance of the class is created in the complete 
execution of a program or project. It is used where only a single instance of a class is required to control the action 
throughout the execution. A singleton class shouldn’t have multiple instances in any case and at any cost. 
Singleton classes are used for logging, driver objects, caching and thread pool, database connections.

Definition: 

The singleton pattern is a design pattern that restricts the instantiation of a class to one object. 

An implementation of singleton class should have following properties:

1. It should have only one instance : This is done by providing an instance of the class from within the class. 
Outer classes or subclasses should be prevented to create the instance. This is done by making the constructor private
in java so that no class can access the constructor and hence cannot instantiate it.
2. Instance should be globally accessible : Instance of singleton class should be globally accessible so that each class
can use it. In Java, it is done by making the access-specifier of instance public.
