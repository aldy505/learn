\documentclass{article}
\usepackage{minted}
\author{Reinaldy Rafli}
\date{\today}
\title{Bitwise Operations}
\begin{document}

\maketitle

In computer programming, a bitwise operation operates on a bit string, a bit array or a binary numeral (considered as a bit string) 
at the level of its individual bits. It is a fast and simple action, basic to the higher level arithmetic operations and directly 
supported by the processor. Most bitwise operations are presented as two-operand instructions where the result replaces one of 
the input operands.

On simple low-cost processors, typically, bitwise operations are substantially faster than division, several times faster 
than multiplication, and sometimes significantly faster than addition. While modern processors usually perform addition 
and multiplication just as fast as bitwise operations due to their longer instruction pipelines and other architectural 
design choices, bitwise operations do commonly use less power because of the reduced use of resources.

From Wikipedia: https://en.wikipedia.org/wiki/Bitwise_operation

\section{Bitwise AND}

The bitwise AND operator (&) returns a 1 in each bit position for which the corresponding bits of both operands are 1s.

\begin{minted}{js}
  const a = 5;        // 00000000000000000000000000000101
  const b = 3;        // 00000000000000000000000000000011

  console.log(a & b); // 00000000000000000000000000000001
  // expected output: 1
\end{minted}

\begin{minted}{}
  .    9 (base 10) = 00000000000000000000000000001001 (base 2)
      14 (base 10) = 00000000000000000000000000001110 (base 2)
                     --------------------------------
  14 | 9 (base 10) = 00000000000000000000000000001111 (base 2) = 15 (base 10)
\end{minted}

\section{Bitwise OR}

The bitwise OR operator (|) returns a 1 in each bit position for which the corresponding bits of either or both operands are 1s.

\begin{minted}{js}
  const a = 5;        // 00000000000000000000000000000101
  const b = 3;        // 00000000000000000000000000000011

  console.log(a | b); // 00000000000000000000000000000111
  // expected output: 7
\end{minted}

\begin{minted}{}
  .    9 (base 10) = 00000000000000000000000000001001 (base 2)
      14 (base 10) = 00000000000000000000000000001110 (base 2)
                     --------------------------------
  14 | 9 (base 10) = 00000000000000000000000000001111 (base 2) = 15 (base 10)
\end{minted}

\section{Bitwise NOT}

The bitwise NOT operator (~) inverts the bits of its operand. Like other bitwise operators, it converts the operand to a 32-bit signed integer

\begin{minted}{js}
  const a = 5;     // 00000000000000000000000000000101
  const b = -3;    // 11111111111111111111111111111101

  console.log(~a); // 11111111111111111111111111111010
  // expected output: -6

  console.log(~b); // 00000000000000000000000000000010
  // expected output: 2
\end{minted}

\begin{minted}{}
   9 (base 10) = 00000000000000000000000000001001 (base 2)
                 --------------------------------
  ~9 (base 10) = 11111111111111111111111111110110 (base 2) = -10 (base 10)
\end{minted}

\section{Bitwise XOR}

The bitwise XOR operator (^) returns a 1 in each bit position for which the corresponding bits of either but not both operands are 1s.

\begin{minted}{js}
const a = 5;        // 00000000000000000000000000000101
const b = 3;        // 00000000000000000000000000000011

console.log(a ^ b); // 00000000000000000000000000000110
// expected output: 6
\end{minted}

\begin{minted}{}
      14 (base 10) = 00000000000000000000000000001110 (base 2)
       9 (base 10) = 00000000000000000000000000001001 (base 2)
                     --------------------------------
  14 ^ 9 (base 10) = 00000000000000000000000000000111 (base 2) = 7 (base 10)
\end{minted}
\end{document}
