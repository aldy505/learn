\documentclass{article}
\usepackage{minted}
\author{Reinaldy Rafli}
\date{\today}
\title{Merge Sort Algorithm}
\begin{document}

\maketitle

The merge sort is a recursive sort of order n*log(n).

It is notable for having a worst case and average complexity of O(n*log(n)),
and a best case complexity of O(n) (for pre-sorted input).

The basic idea is to split the collection into smaller groups by halving it until the groups
only have one element or no elements (which are both entirely sorted groups).

Then merge the groups back together so that their elements are in order.

The merge sort algorithm comes in two parts: sort function and merge function.

\begin{minted}{}
  function mergesort(m)
   var list left, right, result
   if length(m) ≤ 1
       return m
   else
       var middle = length(m) / 2
       for each x in m up to middle - 1
           add x to left
       for each x in m at and after middle
           add x to right
       left = mergesort(left)
       right = mergesort(right)
       if last(left) ≤ first(right) 
          append right to left
          return left
       result = merge(left, right)
       return result

function merge(left,right)
   var list result
   while length(left) > 0 and length(right) > 0
       if first(left) ≤ first(right)
           append first(left) to result
           left = rest(left)
       else
           append first(right) to result
           right = rest(right)
   if length(left) > 0 
       append rest(left) to result
   if length(right) > 0 
       append rest(right) to result
   return result
\end{minted}

\begin{minted}{c}
  #include <stdio.h>
#include <stdlib.h>
 
void merge (int *a, int n, int m) {
    int i, j, k;
    int *x = malloc(n * sizeof (int));
    for (i = 0, j = m, k = 0; k < n; k++) {
        x[k] = j == n      ? a[i++]
             : i == m      ? a[j++]
             : a[j] < a[i] ? a[j++]
             :               a[i++];
    }
    for (i = 0; i < n; i++) {
        a[i] = x[i];
    }
    free(x);
}
 
void merge_sort (int *a, int n) {
    if (n < 2)
        return;
    int m = n / 2;
    merge_sort(a, m);
    merge_sort(a + m, n - m);
    merge(a, n, m);
}
 
int main () {
    int a[] = {4, 65, 2, -31, 0, 99, 2, 83, 782, 1};
    int n = sizeof a / sizeof a[0];
    int i;
    for (i = 0; i < n; i++)
        printf("%d%s", a[i], i == n - 1 ? "\n" : " ");
    merge_sort(a, n);
    for (i = 0; i < n; i++)
        printf("%d%s", a[i], i == n - 1 ? "\n" : " ");
    return 0;
}
\end{minted}

\begin{minted}{go}
  package main
  
  import "fmt"
  
  var a = []int{170, 45, 75, -90, -802, 24, 2, 66}
  var s = make([]int, len(a)/2+1) // scratch space for merge step
  
  func main() {
      fmt.Println("before:", a)
      mergeSort(a)
      fmt.Println("after: ", a)
  }
  
  func mergeSort(a []int) {
      if len(a) < 2 {
          return
      }
      mid := len(a) / 2
      mergeSort(a[:mid])
      mergeSort(a[mid:])
      if a[mid-1] <= a[mid] {
          return
      }
      // merge step, with the copy-half optimization
      copy(s, a[:mid])
      l, r := 0, mid
      for i := 0; ; i++ {
          if s[l] <= a[r] {
              a[i] = s[l]
              l++
              if l == mid {
                  break
              }
          } else {
              a[i] = a[r]
              r++
              if r == len(a) {
                  copy(a[i+1:], s[l:mid])
                  break
              }
          }
      }
      return
  }
\end{minted}

\begin{minted}{js}
  function merge(left, right, arr) {
    var a = 0;
  
    while (left.length && right.length) {
      arr[a++] = (right[0] < left[0]) ? right.shift() : left.shift();
    }
    while (left.length) {
      arr[a++] = left.shift();
    }
    while (right.length) {
      arr[a++] = right.shift();
    }
  }
    
  function mergeSort(arr) {
    var len = arr.length;
  
    if (len === 1) { return; }
  
    var mid = Math.floor(len / 2),
        left = arr.slice(0, mid),
        right = arr.slice(mid);
  
    mergeSort(left);
    mergeSort(right);
    merge(left, right, arr);
  }
  
  var arr = [1, 5, 2, 7, 3, 9, 4, 6, 8];
  mergeSort(arr); // arr will now: 1, 2, 3, 4, 5, 6, 7, 8, 9
\end{minted}